% !TeX spellcheck = ru_RU
% !TEX root = vkr.tex

\section{Реализация}
Очень важный раздел для будущих программных инженеров, т.е. почти для всех. Важно иметь всегда, в том числе для промежуточных отчетов по учебным практикам или ВКР.

В процессе работы можно сделать огромное количество косяков, неполный список которых ниже.

\begin{enumerate}
  \item Реализация должна быть. На публично доступную реализацию обязательная ссылка. Если код под \textsc{NDA}, то об этом, во-первых, должно быть сказано явно, и, во-вторых, на защиту должны выно\-ситься другие результаты (например, архитектура), чтобы комис\-сия имела возможность оценить хоть что-то.
        \begin{itemize}
          \item  Рецензент обязан оценить код (о возможности должен побеспо\-коиться обучающийся).
        \end{itemize}
  \item Код реализации должен быть написан защищающимся целиком.
        \begin{itemize}
          \item  Если проект групповой, то нужно явно выделить какие части были модифицированы защищающимся. Например, в преды\-дущих разделах на картинке архитектуры нужно выделить цветом то, что вы модифицировали.
          \item Нельзя пускать в негрупповой проект коммиты от других людей, или людей не похожих на Вас. Например, в 2022 году защищающийся-парень делал коммиты от сценического псев\-донима, который намекает на женский \enquote{гендер}. (Нет, это не шутка.) На тот момент в российской культуре это выглядело странно.
          \item Возможна ситуация, что вы используете конкретный ник в интернете уже лет пять, и желаете писать ВКР под этим ником на \GitHub{}. В принципе, это допустимо (не только лишь я так считаю), но если Вы встретите преподавателя, который считает наоборот, то Вам придется грамотно отмазы\-ваться. В Вашу пользу могут сыграть те факты, что к нику на гитхабе у Вас приписаны настоящие имя и фамилия; что в репозитории у вас видна домашка за 1й курс; и что Ваш преподаватель практики сможет подтвердить, что Вы уже несколько лет используете это ник; и т.п.
        \end{itemize}
  \item Если вы получаете диплом о присвоении звания программного инженера, код должен соответствовать.
        \begin{enumerate}
          \item Не стоит выкладывать код одним коммитом.
          \item Лучше хоть какие-то тесты, чем совсем без них. В идеале нужно предъявлять процент покрытия кода тестами.
          \item Лучше  сделать \textsc{CI}, а также \textsc{CD}, если оно уместно в Вашем проекте.
          \item Не стоит демонстрировать на защите, что Вам даже не пришло в голову напустить на код линтеры и т.п.
        \end{enumerate}
  \item Если ваша реализация по сути является прохождением стандартного туториала, например, по отделению картинок кружек от котиков с помощью машинного обучения, то необходимо срочно сообщить об этом куратору на мат-мехе, иначе Государственная Экзаменацион\-ная Комиссия \enquote{порвёт Вас как Тузик грелку}, поставит \enquote{единицу}, а все остальные Ваши сокурсники получат оценку выше. (Это не шутка, а реальная история 2020 года.)
\end{enumerate}

\noindent Если Вам предстоит защищать учебную практику, а эти рекомендации видятся как более подходящие для защиты ВКР, то ... отмаза не засчиты\-вается, сразу учитесь делать нормально.
