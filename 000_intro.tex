% !TeX spellcheck = ru_RU
% !TEX root = vkr.tex

\section*{Введение}
\thispagestyle{withCompileDate}
Ныне набирает популярность RISC-V --- расширяемая открытая и свободная
 система команд и
 процессорная архитектура на основе концепции RISC,
  предназначенная для создания процессоров/микроконтроллеров
   и разработки ПО\footnote{\url{https://ru.wikipedia.org/wiki/RISC-V}}.\
   Важную роль в его успехе сыграла открытость кода и свобода использования, что резко отличалось
    от многих других архитектур \footnote[2]{\url{https://habr.com/ru/articles/454208/}}.

	Интринсики (иногда называют встроенными функциями) – это функции,
	 встроенные в компилятор и предназначенные для оптимизации кода на языке ассемблера.
	  Они позволяют разработчикам программного обеспечения использовать специализированные
	   инструкции процессора для ускорения выполнения операций и функций.

Бывают ситуации, когда вызывается функция, которая представляет собой интринсик (т е она повторяет собой поведение интринсика) или
она "легко раскладывается на последовательность инструкций".
Хотелось бы, чтобы компилятор в ассемблер не вызывал эту функцию, а на этом месте сразу же
писал ее ``определение'' или ``тело''. (предположим, у меня есть функция f, я ее вызываю раз 10
до моего вмешательства она вызывалась 10 раз, после него 10 раз в ассемблере будет писаться ее тело
, я правильно понимаю, что это хорошо? ведь после ``ассемблирования'' все вызовы все равно заменятся на тело)

% Тут  4 части (абзаца) максимум на 2 страницы:
% \begin{enumerate}
% \item vvvBackground, known information.
% \item Knowlvasdfedge gap, unknown information.
% \item  Hypothesis, question, purpose statement.
% \item Approach, plan of attack, proposed solution.
% \begin{itemize}
% \item Последний абзац должен читаться и быть понятным в отрыве от остальных трёх.
% \end{itemize}
% \end{enumerate}




% \blfootnote{
% 	Иногда рецензенту полезно знать какого числа компилировался текст, чтобы оценить актуальность версии текста. В этом случае полезно вставлять в текст дату сборки. Для совсем официальных релизов документа это не вполне канон.\\
% Также здесь имеет смысл указать, если работа сделана на деньги, например, Российского Фонда Фундаментальных Исследований (РФФИ) по гранту номер такой-то, и т.п.}
