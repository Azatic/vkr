\section{myfunc}
Пусть в коде встречается функция z = x + y * (2 ** z)
Как реализуется интринзик для этой функции на \textbf{данный момент}:
\begin{enumerate}
    \item x = x/2 //растеггирование x
    \item y = y/2 //растеггирование y
    \item th.addsl z x y (n/2) // вычисление растеггированного z
    \item th.addsl z 1 z 1 //вычисление теггированного z (хотим слева поставить 1) (th.addsl z 1 z 1 ===
    z = 1 + z * 2)
\end{enumerate}
всего 4 инструкции

не исключено, что в лоб легче

\textbf{В лоб}: (допускаем, что 2**z = c, т к z = 0,1,2 и это просто сдвиг)

Имеем z = x + y * z
\begin{enumerate}
    \item y * z = (x >> 1) * (y - 1) 3 операции
    \item x + y*z = x + y*z (без вычета 1, т к не + ее на предыдущем шаге)
\end{enumerate}
Итого 4 инструкции
Непонятно откуда берется выигрыш в первом случае

\textbf{Что можно изменить:}
\begin{enumerate}
    \item в первом подходе не заниматься растегированием x y
    , вычислить z, а потом правильно его протеггировать
    \item рассмотреть ситуацию z = d + x + y*c
\end{enumerate}

пусть у нас встречается ситуация:

\textbf{z = d + x + y*c}:
Первые три шага совпадают с первым случаем
\begin{enumerate}
    \item x = x/2 //растеггирование x
    \item y = y/2 //растеггирование y
    \item th.addsl z x y (n/2) // вычисление растеггированного z
    \item что мы имеем: затеггированный d
    растегированный z.
    если по тупому
    z = th.addsl z 1 z 1 (затегировали z) + d =
    z = $1^c$ + z * 2 + $d^c$ = z * 2 + d = th.addsl z d z 1
\end{enumerate}
Всего также 4 инструкции, несмотря на дополнительное сложение.
